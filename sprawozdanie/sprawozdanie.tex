\documentclass[a4paper,12pt]{mwart}
\usepackage[MeX]{polski}
\usepackage[utf8]{inputenc}
\usepackage{indentfirst}
\usepackage{hyperref}
\usepackage{color}
\usepackage{amsmath}

\usepackage{chngcntr}
\counterwithin*{section}{part}

\frenchspacing

\title{
    Grafy i~Sieci \\
    sprawozdania
}
\author{
    Łukasz~Jędrzejewski
    \and
    Igor~Rodzik
}

\date{}

\begin{document}

\maketitle

\part{Sprawozdanie 1}

\section{Temat projektu}

Graf skierowany $G=(V,E)$ jest częściowo spójny, jeśli dla każdych dwóch
wierzchołków $u$ i $v$ z $V$ istnieje ścieżka $u \to v$ lub $v \to u$. Należy
zaimplementować algorytm sprawdzania, czy dany graf $G$ jest częściowo spójny.

\section{Opis merytoryczny zadania}

Naszym zadaniem jest implementacja algorytmu, który ma~sprawdzić, czy podany
graf skierowany jest częściowo spójny.

\subsection{Definicje}

\begin{description}
\item[Ścieżka] trasa, w~której krawędzie nie powtarzają się.
\item[Trasa] ciąg wierzchołków (gdy graf nie jest multigrafem) lub ciąg
  postaci $(v_1,e_1,v_2,e_2,\ldots, v_{n-1}, e_{n-1}, v_n)$.
\item[Multigraf] graf, dla którego dopuszczamy wielokrotne krawędzie między
  dwoma wierzchołkami oraz pętle.
\item[Pętla] krawędź, której końcami jest ten sam wierzchołek.
\item[Graf częściowo spójny] graf, w~którym między każdą parą wierzchołków
  istnieje ścieżka co~najmniej w~jedną stronę.
\end{description}

\subsection{Założenia}

\begin{enumerate}
\item Rozpatrywany graf jest spójny.
\end{enumerate}

\section{Opis wykorzystywanego algorytmu}

W~algorytmie zamierzamy wykorzystać algorytm przeglądania grafu w~głąb lub
wszerz. Algorytm zarządzałby dwiema kolekcjami wierzchołków -- zbiorem
wierzchołków $V1$, które stanowią podgraf częściowo spójny (na~początku zbiór
pusty) oraz wierzchołkami niesprawdzonymi $V2$ wejściowego grafu (na~początku
wszystkie wierzchołki). Następnie powtarzalibyśmy przechodzenie grafu
pobierając pierwszy wierzchołek z~$V2$. Dla wykonanego przejścia należy
sprawdzić, czy wszystkie odwiedzone wierzchołki zawierają się w~zbiorze $V1$.
Jeśli nie, oznacza to, że~graf nie jest częściowo spójny (ponieważ
z~wierzchołków z~$V1$ nie dotarliśmy do~bieżącego wierzchołka, więc z~tego
wierzchołka musimy osiągnąć wszystkie wierzchołki z~$V1$). Jeśli tak, to~należy
powtórzyć krok, dodając do~$V1$ odwiedzone wierzchołki, do~momentu, gdy zbiór
$V1$, będzie stanowił wszystkie wierzchołki wejściowego grafu, lub nie
osiągniemy odwiedzonych do~tej pory wierzchołków z~$V1$.

\section{Literatura}

\noindent Cormen, Thomas H.; Leiserson, Charles E.; Rivest, Ronald L.; Stein,
\emph{Wprowadzenie do~algorytmów}.

\newpage

\part{Sprawozdanie 2}

\section{Algorytmy}

\subsection{Z~definicji}
\label{sec:from-definion-alg}

Dla każdej pary wierzchołków $(v, u)$ grafu sprawdź, czy można osiągnąć
wierzchołek $u$ uruchamiając procedurę \emph{DFS} z~$v$ lub odwrotnie.

\subsection{Bazujący na~silnie spójnych składowych}

Idea kolejnego algorytmu polega na~wyznaczeniu wszystkich silnie spójnych
składowych grafu. Następnie każdą silnie spójną składową zastępujemy
pojedynczym wierzchołkiem. Składowe łączymy krawędzią, jeśli istnieje krawędź
między wierzchołkami należącymi do~składowych. Na~tak zminimalizowanym grafie,
uruchamiamy algorytm sprawdzający z~sekcji~\ref{sec:from-definion-alg}, czy
graf jest częściowo spójny.

% TODO: przykład transformacji?

\subsubsection{Wyznaczenie silnie spójnych składowych}

Jednym z~algorytmów wyznaczających silnie spójne składowe grafu to~algorytm
\textbf{Kosaraju}. Wykorzystuje on~dwa przejścia w~głąb oraz transpozycję
grafu.

Graf \textbf{transponowany} $G^T$ to~graf, w~którym krawędzie są~odwrócone.

Algorytm zaś wygląda następująco:

\begin{enumerate}
\item\label{sec:kosaraju-alg-1-pt} Wyznaczamy listę wierzchołków w~kolejności
  przetworzenia wierzchołków grafu algorytmem DFS.\@
\item\label{sec:kosaraju-alg-2-pt} Wykonujemy algorytm DFS dla grafu
  transponowanego, dla każdego wierzchołka
  z~punktu~\ref{sec:kosaraju-alg-1-pt}. Każde otrzymane drzewo DFS, zawiera
  wierzchołki należące do~jednej silnie spójnej składowej.
\end{enumerate}

W~punkcie~\ref{sec:kosaraju-alg-1-pt} ważne jest, aby procedurą DFS odwiedzić
wszystkie wierzchołki. W~ten sposób otrzymamy las drzew DFS.\@ Przechodząc
każde drzewo w~kolejności \emph{post-order} (najpierw podrzewa, na~koniec
węzeł), otrzymamy kolejność odwiedzania wierzchołków wymaganą w~kolejnym kroku.

Krok~\ref{sec:kosaraju-alg-2-pt} algorytmu wykorzystuje własność grafu
transponowanego taką, że~posiada on~identyczne silnie spójne składowe, jak graf
pierwotny.

\section{Struktury danych}

\section{Projekt testów}

Obydwa algorytmy przetestujemy na~kilku zdefiniowanych przez nas szczególnych
przypadkach grafów częściowo spójnych, jak i~nie.

Dodatkowo zamierzamy zaimplementować generatory grafów częściowo spójnych i~nie
posiadających tej własności, w~celu sprawdzenia poprawności działania
na~losowych grafach. Wytworzony generator posłuży nam także do~generacji
większych grafów oraz weryfikacji złożoności czasowych algorytmów.

\section{Założenia programu}

\end{document}

%%% Local Variables:
%%% mode: latex
%%% TeX-master: t
%%% End:
