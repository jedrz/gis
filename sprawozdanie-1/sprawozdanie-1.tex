\documentclass[a4paper,12pt]{mwart}
\usepackage[MeX]{polski}
\usepackage[utf8]{inputenc}
\usepackage{indentfirst}
\usepackage{hyperref}
\usepackage{color}
\usepackage{amsmath}
\frenchspacing

\title{
    Grafy i~Sieci \\
    uszczegółowienie tematu
}
\author{
    Łukasz~Jędrzejewski
    \and
    Igor~Rodzik
}

\date{}

\begin{document}

\maketitle

\section{Temat projektu}

Graf skierowany $G=(V,E)$ jest częściowo spójny, jeśli dla każdych dwóch
wierzchołków $u$ i $v$ z $V$ istnieje ścieżka $u \to v$ lub $v \to u$. Należy
zaimplementować algorytm sprawdzania, czy dany graf $G$ jest częściowo spójny.

\section{Opis merytoryczny zadania}

Naszym zadaniem jest implementacja algorytmu, który ma~sprawdzić, czy podany
graf skierowany jest częściowo spójny.

\subsection{Definicje}

\begin{description}
\item[Ścieżka] trasa, w~której krawędzie nie powtarzają się.
\item[Trasa] ciąg wierzchołków (gdy graf nie jest multigrafem) lub ciąg
  postaci $(v_1,e_1,v_2,e_2,\ldots, v_{n-1}, e_{n-1}, v_n)$.
\item[Multigraf] graf, dla którego dopuszczamy wielokrotne krawędzie między
  dwoma wierzchołkami oraz pętle.
\item[Pętla] krawędź, której końcami jest ten sam wierzchołek.
\item[Graf częściowo spójny] graf, w~którym między każdą parą wierzchołków
  istnieje ścieżka co~najmniej w~jedną stronę.
\end{description}

\subsection{Założenia}

\begin{enumerate}
	\item Rozpatrywany graf jest spójny.
\end{enumerate}

\section{Opis wykorzystywanego algorytmu}

W~algorytmie zamierzamy wykorzystać algorytm przeglądania grafu w~głąb lub
wszerz. Algorytm zarządzałby dwiema kolekcjami wierzchołków -- zbiorem
wierzchołków $V1$, które stanowią podgraf częściowo spójny (na~początku zbiór
pusty) oraz wierzchołkami niesprawdzonymi $V2$ wejściowego grafu (na~początku
wszystkie wierzchołki). Następnie powtarzalibyśmy przechodzenie grafu
pobierając pierwszy wierzchołek z~$V2$. Dla wykonanego przejścia należy
sprawdzić, czy wszystkie odwiedzone wierzchołki zawierają się w~zbiorze $V1$.
Jeśli nie, oznacza to, że~graf nie jest częściowo spójny (ponieważ
z~wierzchołków z~$V1$ nie dotarliśmy do~bieżącego wierzchołka, więc z~tego
wierzchołka musimy osiągnąć wszystkie wierzchołki z~$V1$). Jeśli tak, to~należy
powtórzyć krok, dodając do~$V1$ odwiedzone wierzchołki, do~momentu, gdy zbiór
$V1$, będzie stanowił wszystkie wierzchołki wejściowego grafu, lub nie
osiągniemy odwiedzonych do~tej pory wierzchołków z~$V1$.

\section{Literatura}


\end{document}

%%% Local Variables:
%%% mode: latex
%%% TeX-master: t
%%% End:
